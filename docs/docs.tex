\documentclass{book}

\title{HLIM Language Manual}
\date{\today}
\author{Hourin}

\begin{document}

\maketitle
\tableofcontents

\frontmatter
\chapter{Introduction}
The HLIM is an interpreted programming language designed with a focus on
functional programming principles and syntactic minimalism. An environment
where gotchas and bugs happen due to syntactic gotchas is provided by reducing
the number of unique syntactic constructs.

The language, and by extension its first interpreter, is dynamic and
weakly typed. Types are bound to values at runtime rather than declared
identifiers. This is one of the biggest differences between HLIM and
conventional functional languages. This removes overhead for the interpreter
and the interpreter developer.

Branching via \texttt{if()} and looping via \texttt{loop()} is provided.
Early loop exiting can be done by forcefully making the loop condition
false. I reckon the inclusion of a dedicated exit statement would be great,
but it might clutter the control flow.

Having first\-class functions, HLIM allows the user to freely pass a
function as an argument for other functions, and function to return
another function in order to compose bigger functions, and make the
functions more flexible. However, even though most of a higher\-order
function/closure's functions are implemented including non\-local variables
to some extents, capturing specific variables like in C\+\+ or Rust is not
yet implemented.

\mainmatter
\chapter{Tutorial Introduction}
\section{Hello World}

\end{document}

